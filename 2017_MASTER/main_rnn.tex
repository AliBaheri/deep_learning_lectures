\documentclass[aspectratio=169]{beamer}
%
% Choose how your presentation looks.
%
% For more themes, color themes and font themes, see:
% http://deic.uab.es/~iblanes/beamer_gallery/index_by_theme.html
%
\mode<presentation>
{
  \usetheme{metropolis}      % or try Darmstadt, Madrid, Warsaw, ...
  \usecolortheme{default} % or try albatross, beaver, crane, ...
  \usefonttheme{structurebold}  % or try serif, structurebold, ...
  \setbeamercolor{background canvas}{bg=white}
  \setbeamertemplate{navigation symbols}{}
  \setbeamertemplate{bibliography item}{\insertbiblabel}
  %\setbeamertemplate{caption}[numbered]
} 
\usepackage[english]{babel}
\usepackage[utf8x]{inputenc}
\usepackage{listings}             % Include the listings-package
\hypersetup{
    colorlinks = true,
    linkcolor = {black},
    urlcolor = {blue}
}

\DeclareMathOperator*{\argmin}{arg\,min}

\title[Deep Learning and Temporal Data Processing]{Deep Learning and Temporal Data Processing}
\subtitle{3 - Recurrent Neural Networks}
\institute{University of Modena and Reggio Emilia}
\author{Andrea Palazzi}
\date{June 21th, 2017}

\def\thisframelogos{}

\newcommand{\framelogo}[1]{\def\thisframelogos{#1}}

\addtobeamertemplate{frametitle}{}{%
\begin{tikzpicture}[remember picture,overlay]
\node[anchor=north east] at (current page.north east) {%
    \foreach \img in \thisframelogos {%
        %\hspace{.5ex}%
        \includegraphics[height=3.5ex]{\img}%
    }%
};
\end{tikzpicture}}

\begin{document}

\framelogo{logo_unimore_white.png}

\bgroup
\renewcommand{\insertframenumber}{}
\begin{frame}[noframenumbering]
  \titlepage
\end{frame}
\egroup
\begin{frame}{Agenda}
  \tableofcontents
\end{frame}


%%%%%%%%%%%%%%%%%%%%%%%%%%%%%%%%%%%%%%%%%%%%%%%%%%%%%%%%%%%%%%%%%%
%%%%%%%%%%%%%%%%%%%%%%%%%%%%%%%%%%%%%%%%%%%%%%%%%%%%%%%%%%%%%%%%%%
%%%%%%%%%%%%%%%%%%%%%%%%%%%%%%%%%%%%%%%%%%%%%%%%%%%%%%%%%%%%%%%%%%

\section{Introduction}

%%%%%%%%%%%%%%%%%%%%%%%%%%%%%%%%%%%%%%%%%%%%%%%%%%%%%%%%%%%%%%%%%%

\begin{frame}{Recurrent Neural Networks: overview}
\cite{yu2015multi}
\end{frame}

%%%%%%%%%%%%%%%%%%%%%%%%%%%%%%%%%%%%%%%%%%%%%%%%%%%%%%%%%%%%%%%%%%

\begin{frame}{Notes}
No matter how's the network topology, during backpropagation the network is unfolded in a DAG, so there are no loops.
\end{frame}

%%%%%%%%%%%%%%%%%%%%%%%%%%%%%%%%%%%%%%%%%%%%%%%%%%%%%%%%%%%%%%%%%%
%%%%%%%%%%%%%%%%%%%%%%%%%%%%%%%%%%%%%%%%%%%%%%%%%%%%%%%%%%%%%%%%%%
%%%%%%%%%%%%%%%%%%%%%%%%%%%%%%%%%%%%%%%%%%%%%%%%%%%%%%%%%%%%%%%%%%

\begin{frame}[t]
\frametitle{References}
\bibliographystyle{abbrv}
\bibliography{bibliography}
\end{frame}
\end{document}